% LaTeX resume using res.cls
\documentclass[line,10pt]{res} 
% \usepackage{helvetica} % uses helvetica postscript font (download helvetica.sty)
%\usepackage{newcent}   % uses new century schoolbook postscript font 
\usepackage{enumitem}
\usepackage{hyperref}
\usepackage{xcolor}
\hypersetup{
  colorlinks=true,
  linkcolor=blue,
  linkbordercolor=blue,
  urlcolor=blue
}
\newcommand{\resumetitlemar}{0.5em}
\newsectionwidth{1em}

\begin{document}

\name{\huge{\sc \textbf Francis Williams}}
\address{francis@fwilliams.info}
\address{+1-650-701-7891}

\begin{resume}

\vspace{0.2em}
\section{\Large \sc \textbf Summary}
\vspace{0.5em}
I am a PhD student in the \href{https://cims.nyu.edu/gcl/}{Geometric Computing Lab} at NYU. I am interested in problems in machine learning and geometry. In particular, my research aims at designing, developing, and understanding algorithms to process and understand geometric data acquired from scans of the real world. In addition to research, I have a strong background in software engineering and systems programming backed by several years of industry experience.
\vspace{0.5em}





\vspace{0.2em}
\section{\Large \sc \textbf Education}
\vspace{0.5em}

{\sl \textbf{PhD in Computer Science (In Progress, Year 2)}} - GPA: 3.967/4.00 \\ 
New York University, New York, NY \\
Advisors: \href{https://cims.nyu.edu/gcl/daniele.html}{Daniele Panozzo} and \href{https://vgc.poly.edu/~csilva/}{Claudio Silva}

{\sl \textbf{Bachelor of Software Engineering - Honors Co-operative program}} - GPA: 81\% (A)\\ 
University of Waterloo, ON, Canada \\
Graduated with Distinction \\
October $1^{st}$ 2015




\vspace{0.2em}
\section{\Large \sc \textbf Publications}
\vspace{0.5em}

{\sl \textbf{\href{https://arxiv.org/abs/1811.10943}{Deep Geometric Prior for Surface Reconstruction}} - CVPR 2019}\\ 
\textbf{Francis Williams}, Teseo Schneider, Claudio Silva, Denis Zorin, Joan Bruna, Daniele Panozzo

{\sl \textbf{\href{https://arxiv.org/abs/1812.06216}{ABC: A Big CAD Model Dataset For Geometric Deep Learning}} - CVPR 2019}\\ 
Sebastian Koch, Albert Matveev, Zhongshi Jiang, \textbf{Francis Williams}, Alexey Artemov, Evgeny Burnaev, Marc Alexa, Denis Zorin, Daniele Panozzo

{\sl \textbf{\href{https://arxiv.org/abs/1904.04890}{Unwind: Interactive Fish Straightening}} - Submitted to IEEE VIS 2019}\\
\textbf{Francis Williams}, Alexander Bock, Harish Doraiswamy, Cassandra Donatelli, Adam Summers, Daniele Panozzo, Claudio Silva




\vspace{0.2em}
\section{\Large \sc \textbf Open Source}
\vspace{0.5em}

{\sl \textbf{NumpyEigen}}\\
\url{https://github.com/fwilliams/numpyeigen}
\vspace{\resumetitlemar}
\begin{itemize} \itemsep -2pt
\item A library for fast zero-overhead bindings between Numpy and Eigen
\vspace{-0.4em}
\begin{itemize} \itemsep -2pt
\item Makes it easy to transparently convert Numpy dense and Sparse arrays into Eigen types while taking full advantage of expression template optimizations in Eigen
\item Features near-zero performance overhead and supports function overloading
\item Used by LibIGL, a major open source project, for Python bindings 
\end{itemize}
\end{itemize}

{\sl \textbf{Point Cloud Utils}}\\
\url{https://github.com/fwilliams/point-cloud-utils}
\vspace{\resumetitlemar}
\begin{itemize} \itemsep -2pt
\item A Python utility library exposing common algorithms on 3D point clouds and meshes
\vspace{-0.4em}
\begin{itemize} \itemsep -2pt
\item Random point sampling of meshes with Poisson Disk Sampling and Lloyd-Relaxation.
\item Fast pairwise nearest neighbor.
\item Point set distances including Chamfer, Sinkhorn, and Hausdorff.
\end{itemize}
\end{itemize}

{\sl \textbf{FML - Francis' Machine-Learnin' Library}}\\
\url{https://github.com/fwilliams/fml}
\vspace{\resumetitlemar}
\begin{itemize} \itemsep -2pt
\item A collection of Pytorch utilities for machine learning tasks
\vspace{-0.4em}
\begin{itemize} \itemsep -2pt
\item Includes a numerically stable implementation of the Sinkhorn algorithm to compute optimal transport of point sets in any dimension.
\item Also includes a vectorized implementation of the Chamfer Distance between point sets in any dimension.
\end{itemize}
\end{itemize}

{\sl \textbf{Unwind}}\\
\url{https://github.com/fwilliams/unwind}
\vspace{\resumetitlemar}
\begin{itemize} \itemsep -2pt
\item A software tool for segmenting and unwarping volumetric CT scans of fishes
\vspace{-0.4em}
\begin{itemize} \itemsep -2pt
\item Currently deployed in 2 Labs (at Tufts and the University of Washington) with plans for expansion
\item \href{https://arxiv.org/abs/1904.04890}{Paper} in submission to Vis 2019
\end{itemize}
\end{itemize}

{\sl \textbf{LibIGL}}\\
\url{https://github.com/fwilliams/libigl}
\vspace{\resumetitlemar}
\begin{itemize} \itemsep -2pt
\item I actively contribute to LibIGL, an open source geometry processing library
\vspace{-0.4em}
\begin{itemize} \itemsep -2pt
\item Designed, wrote, and maintained new Python bindings (release planned for July)
\item Implemented techniques for meshing 
\item Implemented volume rendering in the viewer
\end{itemize}
\end{itemize}





\vspace{0.2em}
\section{\Large \sc \textbf Work Experience}
\vspace{0.5em}

{\sl \textbf{Software Engineer}} \hfill Jan 2016. - Jan. 2017 \\
MemSQL \hfill San Francisco, CA\null
\vspace{\resumetitlemar}
\begin{itemize} \itemsep -2pt
\item Developed \href{https://docs.memsql.com/operational-manual/v5.7/role-based-access-control-rbac-deployment-guide/}{Role-Based Access Control} into the MemSQL database engine.
\item As part of a team of two, designed and implemented \href{https://docs.memsql.com/memsql-pipelines/v6.0/pipelines-overview/}{Pipelines}, a high throughput distributed data ingest and transformation engine. 
\item Designed and implemented a subprocess management system in the database engine allowing MemSQL to execute external code in a secure manner without leaking resources.
\item Designed and implemented a general database backup system which allowed users to specify a backup target (e.g. Amazon S3) and automatically write backups and snapshots to this location. 
\end{itemize}

{\sl \textbf{Research Intern}} \hfill Sept. - Dec. 2014 \\
HP Labs, Systems Software Group \hfill Palo Alto, CA\null
\vspace{\resumetitlemar}
\begin{itemize}  \itemsep -2pt
\item Under supervision of Dr. Joseph Tucek, researched designed and implemented distributed file system to run on a simulated memristor computer with $\sim$10 TiB of DRAM connected to hundreds of CPUs with the goal of evaluating performance of file workloads in persistent memory computing environment
\item Independently designed software architecture by reading existing file system papers and source code
\item Designed and implemented directory system, file descriptor management system, journal, and preliminary file data structures
\item Delivered functional prototype with demo which compiled source code stored in the file system using \textit{Make}
\item Mentored junior engineering intern throughout project
\end{itemize}

\pagebreak

{\sl \textbf{Research Assistant}} \hfill Jan. 2014 - Apr. 2014\null\\
Oregon State University \hfill Corvallis, OR\null
\vspace{\resumetitlemar}
\begin{itemize}  \itemsep -2pt
\item In collaboration with Dr. Eugene Zhang, researched and developed a correct and efficient algorithm to compute photorealistic lighting in 3D scenes containing multiple interacting mirror surfaces
\item Formulated theory to model lighting conditions due to mirrors by understanding how mirrors change the topology of the underlying path space of a scene
\item Generated results demonstrating the quality of our method compared to existing techniques
\item Designed and developed algorithm as a modification of \href{http://www.povray.org}{\textit{POV-Ray}}, an existing open source ray tracer
\item Independently learned about many new subjects to conduct research, including group theory, topology, and photorealistic rendering
\end{itemize}

{\sl \textbf{Graphics Software Engineering Intern}} \hfill May - Aug. 2013 \\
Amazon (A9.com) \hfill Palo Alto, CA\null
\vspace{\resumetitlemar}
\begin{itemize}  \itemsep -2pt
\item Independently developed cross platform graphics framework for rendering 3D scans of products
\item The framework supported model and image based rendering techniques with the ability to easily add new rendering modes
\item The framework could be deployed to Linux, Windows, Android, iOS and any WebGL compatible browser from a single C++ code-base using \href{https://github.com/kripken/emscripten}{\textit{emscripten}}
\end{itemize}

{\sl \textbf{Augmented Reality Intern}} \hfill Sept. - Dec. 2012 \\
NVIDIA \hfill Durham, NC\null
\vspace{\resumetitlemar}
\begin{itemize} \itemsep -2pt
\item Designed, improved and debugged real time vision based orientation tracking algorithms
\item Developed Android multimedia framework to decode and playback synchronized video data into Augmented Reality engine
\item Developed Unity3D plugin to stream video data to a texture on Android
\item Built series of Unity3D plug-ins to develop Augmented Reality applications in Unity
\end{itemize}




\end{resume}
\end{document}
